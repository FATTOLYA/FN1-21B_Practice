\documentclass[12pt]{article}
\usepackage[utf8]{inputenc}
\usepackage[russian]{babel}
\usepackage{amsmath,amssymb}
\usepackage{graphics}
\usepackage{pbox}
\usepackage[x11names]{xcolor}
\definecolor{brightmaroon}{rgb}{0.76, 0.13, 0.28}
\definecolor{royalazure}{rgb}{0.0, 0.22, 0.66}
\usepackage[colorlinks=true,linkcolor=royalazure]{hyperref}
\usepackage{tikz, tkz-fct, pgfplots}
\usetikzlibrary{arrows}
\usepackage{geometry}
\geometry{
	a4paper,
	total={170mm,257mm},
	left=20mm,
	top=20mm
} 
\usepackage[labelsep=period]{caption}

% ----------------- Commands ----------------- 
\newcommand{\eps}{\varepsilon}
\newcommand\tline[2]{$\underset{\text{#1}}{\text{\underline{\hspace{#2}}}}$}

% ----------------- Set graphics path ----------------- 
\graphicspath{{img/}}
\begin{document}
\pagestyle{empty}
\centerline{\large Министерство науки и высшего образования}	
\centerline{\large Федеральное государственное бюджетное образовательное}
\centerline{\large учреждение высшего образования}
\centerline{\large ``Московский государственный технический университет}
\centerline{\large имени Н.Э. Баумана}
\centerline{\large (национальный исследовательский университет)''}
\centerline{\large (МГТУ им. Н.Э. Баумана)}
\hrule
\vspace{0.5cm}
\begin{figure}[h]
\center
\includegraphics[height=0.35\linewidth]{bmstu-logo-small.png}
\end{figure}
\begin{center}
	\large	
	\begin{tabular}{c}
		Факультет ``Фундаментальные науки'' \\
		Кафедра ``Высшая математика''		
	\end{tabular}
\end{center}
\vspace{0.5cm}
\begin{center}
	\LARGE \bf	
	\begin{tabular}{c}
		\textsc{Отчёт} \\
		по учебной практике \\
		за 1 семестр 2020---2021 гг.
	\end{tabular}
\end{center}
\vspace{0.5cm}
\begin{center}
	\large
	\begin{tabular}{p{5.3cm}ll}
		\pbox{5.45cm}{
			Руководитель практики,\\
			ст. преп. кафедры ФН1} 	& \tline{\it(подпись)}{5cm} & Кравченко О.В. \\[0.5cm]
		студент группы ФН1--11 		& \tline{\it(подпись)}{5cm} & Ф.И.О.
	\end{tabular}
\end{center}
\vfill
\begin{center}
	\large	
	\begin{tabular}{c}
		Москва, 
		2020 г.
	\end{tabular}
\end{center}

\newpage	
\tableofcontents

\newpage
\section{Цели и задачи практики}	
\subsection{Цели}
--- развитие компетенций, способствующих успешному освоению материала бакалавриата и необходимых в будущей профессиональной деятельности.

\subsection{Задачи}
\begin{enumerate}
\item Знакомство с программными средствами, необходимыми в будущей профессиональной деятельности.
\item Развитие умения поиска необходимой информации в специальной литературе и других источниках.
\item Развитие навыков составления отчётов и презентации результатов.
\end{enumerate}

\subsection{Индивидуальное задание}	
\begin{enumerate}
\item Изучить способы отображения математической информации в системе вёртски \LaTeX.
\item Изучить возможности  системы контроля версий \textsf{Git}.
\item Научиться верстать математические тексты, содержащие формулы и графики в системе \LaTeX.
Для этого, выполнить установку свободно распространяемого дистрибутива \textsf{TeXLive} и оболочки \textsf{TeXStudio}.
\item Оформить в системе \LaTeX типовые расчёты по курсе математического анализа согласно своему варианту.
\item Создать аккаунт на онлайн ресурсе \textsf{GitHub} и загрузить исходные \textsf{tex}--файлы 
и результат компиляции в формате \textsf{pdf}.
\end{enumerate} 

\newpage
\section{Отчёт}
Актуальность темы продиктована необходимостью владеть системой вёрстки \LaTeX и средой вёрстки \textsf{TeXStudio} для
отображения текста, формул и графиков. Полученные в ходе практики навыки могут быть применены при написании
курсовых проектов и дипломной работы, а также в дальнейшей профессиональной деятельности.

Ситема вёрстки \LaTeX содержит большое количество инструментов (пакетов), упрощающих отображение информации в различных 
сферах инженерной и научной деятельности. 

\newpage
\section{Индивидуальное задание}
\subsection{Пределы и непрерывность.}
% ---------------------------- Problem 1----------------------------------
\subsubsection*{\center Задача № 1.}
{\bf Условие.~}
Дана последовательность $a_{n}=\dfrac{5n+15}{6-n}$ и число $c=-5$.
Доказать, что $$\lim\limits_{n\rightarrow\infty}a_n=c,$$
а именно, для каждого $\eps>0$ найти наименьшее натуральное число $N{=}N(\eps)$ такое, что $|a_{n}-c|<\eps$ для всех $n>N(\eps)$.
Заполнить таблицу:
\begin{center}
\begin{tabular}{ | p{25pt} | c | c | c | c |}
\hline
$\varepsilon$& $0{,}1$ & $0{,}01$ & $0{,}001$ \\
\hline
$N(\varepsilon)$ &   &   &\\
\hline
\end{tabular}    
\end{center}
{\bf Решение.~}
Рассмотрим неравенство $a_n-c<\eps,\,\forall\eps>0$, учитывая выражение для $a_n$ и значение $c$ из условия варианта,
получим:
$$
\biggl|\dfrac{5n+15}{6-n}+5\biggr| < \eps ;
$$
$$
\biggl|\frac{5n+15+30-5n}{6-n}\biggr| < \eps ;
$$
$$
\biggl|\frac{45}{6-n}\biggr| < \eps .
$$
Заметим, что $n\in\mathbb{N}$ и $\eps >0$ поэтому, можно опустить знак модуля:
$$
\begin{array}{c}
\dfrac{45}{6-n} < \eps ;                         \\[8pt]
n > \dfrac{45}{\eps} + 6 .                       \\[8pt]
\end{array}
$$
\center Заполним таблицу:
\begin{center}
\begin{tabular}{ | p{25pt} | c | c | c | c |}
\hline
$\varepsilon$& $0{,}1$ & $0{,}01$ & $0{,}001$ \\
\hline
$N(\varepsilon)$& $456$ & $4506$ & $45006$\\
\hline
\end{tabular}
\end{center} 
% ---------------------------- Problem 2----------------------------------
\subsubsection*{\center Задача № 2.}
{\bf Условие.~}
Вычислить пределы функций
$$
\begin{array}{cc}
\text{\bf(а):} & \lim\limits_{x\rightarrow2}\dfrac{x^3-5x+2}{x^2-12x+20}; \\[10pt]
\text{\bf(б):} & \lim\limits_{x\rightarrow\infty}\dfrac{x^3\sqrt{\,x^8+9x^5}+x^4}{3x^4-2x^3}; \\[10pt]
\text{\bf(в):} & \lim\limits_{x\rightarrow\infty}\dfrac{\sqrt[4]{\,x}-1}{\sqrt[3]{\,x}-1}; \\[10pt]
\text{\bf(г):} & \lim\limits_{x\rightarrow3/2}\biggl(1-\dfrac{2x}{3}\biggr)^{\tan{\frac{\pi x}{3}}}; \\[10pt]
\text{\bf(д):} & \lim\limits_{x\rightarrow0}\biggl(\dfrac{\sin{2x}}{\lg(x+1)}\biggr)^{\frac{2}{x+\cos{x}}}; \\[10pt]
\text{\bf(е):} & \lim\limits_{x\rightarrow0}\dfrac{1-\cos{5x}}{e^{x^2}-1}.
\end{array}
$$
% ---------------------------- Problem 2а --------------------------------
{\bf Решение:~}
\text{\bf(а):}
$$
\begin{array}{l}
\lim\limits_{x\rightarrow2}\dfrac{x^3-5x+2}{x^2-12x+20} = 
\lim\limits_{x\rightarrow2}\dfrac{(x-2)(x^2+2x-1)}{(x-2)(x-10)} = 
\lim\limits_{x\rightarrow2}\dfrac{x^2+2x-1}{x-10} = 
-\dfrac{7}{8}.
\end{array}
$$
% ---------------------------- Problem 2б --------------------------------
\text{\bf(б):}
$$
\begin{array}{l}
\lim\limits_{x\rightarrow\infty}\dfrac{x^3\sqrt{\,x^8+9x^5}+x^4}{3x^4-2x^3} =
\lim\limits_{x\rightarrow\infty}\dfrac{\frac{1}{x}+\sqrt{\,1+\frac{9}{x^3}}+1)}{3-\frac{2}{x}} = \dfrac{2}{3}.
\end{array}
$$
% ---------------------------- Problem 2в --------------------------------
\text{\bf(в):}
$$
\begin{array}{l}
\lim\limits_{x\rightarrow\infty}\dfrac{\sqrt[4]{\,x}-1}{\sqrt[3]{\,x}-1} = 
\biggl|
\begin{array}{l}
x=y^{12}\\ y\rightarrow\infty
\end{array}
\biggr|
= \lim\limits_{y\rightarrow\infty}\dfrac{\sqrt[4]{\,y^{12}-1}}{\sqrt[3]{\,y^{12}-1}} = 
\lim\limits_{y\rightarrow\infty}\dfrac{y^{12}-1}{y^{12}-1} = \lim\limits_{y\rightarrow\infty}\dfrac{\frac{y^3}{y^4}-\frac{1}{y^4}}{\frac{y^4}{y^4}-\frac{1}{y}} = 0.
\end{array}
$$
% ---------------------------- Problem 2г --------------------------------
\text{\bf(г):}	
$$
\begin{array}{l}
\lim\limits_{x\rightarrow3/2}\biggl(1-\dfrac{2x}{3}\biggr)^{\tan{\frac{\pi x}{3}}} = 
\biggl|
\begin{array}{l}
t = x-\frac{3}{2} \\ t\rightarrow0
\end{array}
\biggr| =
\lim\limits_{t\rightarrow0}\biggl(1-\dfrac{2(t-\frac{3}{2})}{3}\biggr)^{\tan{\frac{\pi (t+\frac{3}{2})}{3}}} =
\lim\limits_{t\rightarrow0}\biggl(\biggl(1-\dfrac{2t}{3}\biggr)^{\frac{3}{2t}}\biggr)^{\frac{2t\cos{\frac{\pi t}{3}}}{3(-\sin{\frac{\pi t}{3}})}} = \\ 
\lim\limits_{t\rightarrow0}e^{\frac{2t\cos{\frac{\pi t}{3}}\frac{\pi t}{3}\frac{3}{\pi t}}{-3\sin{\frac{\pi t}{3}}}} = \lim\limits_{t\rightarrow0}e^{-\frac{2}{\pi}} = e^{-\frac{2}{\pi}}.
\end{array}
$$
% ---------------------------- Problem 2д --------------------------------
\text{\bf(д):}
$$
\begin{array}{l}
\lim\limits_{x\rightarrow0}\biggl(\dfrac{\sin{2x}}{\lg(x+1)}\biggr)^{\frac{2}{x+\cos{x}}} = 
\lim\limits_{x\rightarrow0}\biggl(\dfrac{\sin{2x}}{\frac{x}{\ln{10}}}\biggr)^{\frac{2}{x+\cos{x}}} =
\lim\limits_{x\rightarrow0}\biggl(\dfrac{\sin{2x}\cdot2\ln{10}}{2-x}\biggr)^{\frac{2}{x+\cos{x}}} = \\
\lim\limits_{x\rightarrow0}(2\ln{10})^{\frac{2}{x+\cos{x}}} = 4\cdot(\ln{10})^2.
\end{array}
$$
% ---------------------------- Problem 2е --------------------------------
\text{\bf(е):}
$$
\begin{array}{l}
\lim\limits_{x\rightarrow0}\dfrac{1-\cos{5x}}{e^{x^2}-1} = \lim\limits_{x\rightarrow0}\dfrac{\frac{(5x)^2}{2}}{x^2} = \dfrac{25}{2}
\end{array}
$$
\newpage
% ---------------------------- Problem 3----------------------------------
\subsubsection*{\center Задача № 3.}
{\bf Условие.~}\\
\text{\bf(а):} Показать, что данные функции
$f(x)$ и $g(x)$ являются бесконечно малыми или бесконечно большими
при указанном стремлении аргумента. \\
\text{\bf(б):} Для каждой функции $f(x)$ и $g(x)$ записать главную часть
(эквивалентную ей функцию)  вида $C(x-x_0)^{\alpha}$ при $x\rightarrow x_0$ или $Cx^{\alpha}$
при $x\rightarrow\infty$, указать их порядки малости (роста). \\
\text{\bf(в):} Сравнить функции $f(x)$ и $g(x)$ при указанном стремлении.
\begin{center}
	\begin{tabular}{|c|c|c|}
		\hline
		№ варианта & функции $f(x)$ и $g(x)$ & стремление \\[6pt]
		%\hline
		18 & $f(x) = \sin{\frac{1}{\sqrt{\,x+\sqrt{\,x}+1}}},~g(x)=\sqrt{\,x^2+\sqrt{\,x}}-x$ & $x\rightarrow+\infty$ \\
		\hline
	\end{tabular}
\end{center}
{\bf Решение.~}\\
\text{\bf(а):}~Покажем, что $f(x)$ и $g(x)$ бесконечно малые функции:
$$
\begin{array}{cc}
\lim\limits_{x\rightarrow+\infty}f(x) = \lim\limits_{x\rightarrow+\infty}\biggl(\sin{\frac{1}{\sqrt{\,x+\sqrt{\,x}+1}}}\biggr) \thicksim \lim\limits_{x\rightarrow+\infty}\biggl(\frac{1}{\sqrt{\,x+\sqrt{\,x}+1}}\biggr) = \lim\limits_{x\rightarrow+\infty}\biggl(\frac{\frac{1}{\sqrt{x}}}{\sqrt{\,\frac{x}{x}+\frac{\sqrt{x}}{x}}+\frac{1}{\sqrt{x}}}\biggr) = 0.
\\
\lim\limits_{x\rightarrow+\infty}g(x) = \lim\limits_{x\rightarrow+\infty}\biggl(\sqrt{\,x^2+\sqrt{\,x}}-x\biggr) = 
\lim\limits_{x\rightarrow+\infty}\biggl(\frac{x^2+\sqrt{\,x}-x^2}{\sqrt{\,x^2+\sqrt{\,x}}+x}\biggr) = 
\lim\limits_{x\rightarrow+\infty}\biggl(\frac{\sqrt{\,x}}{\sqrt{\,\frac{x^2}{x^2}+\frac{\sqrt{\,x}}{x^2}}+\frac{x}{x}}\biggr) = 0.
\end{array}
$$
\text{\bf(б):}~Выделим главные части функций $f(x)$ и $g(x)$:
$$ 
f(x) = \sin{\frac{1}{\sqrt{\,x+\sqrt{\,x}+1}}} \thicksim \frac{1}{\sqrt{\,x}} .
$$
\text{\bf}~Тогда при $x\rightarrow+\infty$ главная часть функции будет $1\cdot x^{1/2}$ .
$$ 
g(x) = \sqrt{\,x^2+\sqrt{\,x}}-x = \dfrac{x^2+\sqrt{\,x}-x^2}{\sqrt{\,x^2+\sqrt{\,x}}+x} = \dfrac{\sqrt{\,x}}{\sqrt{\,x^2+\sqrt{\,x}}+x} = \dfrac{\frac{\sqrt{\,x}}{x}}{\sqrt{\,\frac{x^2}{x^2}+\frac{\sqrt{\,x}}{x^2}}+\frac{x}{x}} = \\
= \dfrac{\frac{x}{\sqrt{\,x}}}{\sqrt{\,1+\frac{1}{\sqrt{\,x^3}}}+1} \thicksim \dfrac{\frac{1}{\sqrt{x}}}{2} = \dfrac{1}{2\sqrt{x}} . \\
$$
\text{\bf}~Тогда при $x\rightarrow+\infty$ главная часть функции будет $\dfrac{1}{2}\cdot x^{1/2}$ . \\
\text{\bf}~$k_f = -\frac{1}{2}$ - порядок малости БМФ $f(x)$ относительно $x\rightarrow+\infty$ . \\
$k_g = -\frac{1}{2}$ - порядок малости БМФ $g(x)$ относительно $x\rightarrow+\infty$ .
\newpage
\text{\bf(в):}~Для сравнения функций $f(x)$ и $g(x)$ рассмотрим предел их отношения при указанном стремлении
$$
\lim\limits_{x\rightarrow+\infty}\dfrac{f(x)}{g(x)}.
$$
Применим эквивалентности, определенные в пункте (б), получим
$$
\lim\limits_{x\rightarrow+\infty}\dfrac{f(x)}{g(x)} = 
\lim\limits_{x\rightarrow+\infty}\dfrac{\sin{\frac{1}{\sqrt{\,x+\sqrt{\,x}+1}}}}{\sqrt{\,x^2+\sqrt{\,x}}-x} =
\lim\limits_{x\rightarrow+\infty}\dfrac{\frac{1}{\sqrt{\,x+\sqrt{\,x}+1}}}{\sqrt{\,x^2+\sqrt{\,x}}-x} = 2 .
$$
Отсюда следует, что $f(x)$ и $g(x)$ -- функции одного порядка.
%=========================================================================
\newpage
\addcontentsline{toc}{section}{Список литературы}
\begin{thebibliography}{99}
\bibitem{book01} Львовский С.М. Набор и вёрстка в системе \LaTeX, 2003 c.
\bibitem{book02} Котельников И.А., Чеботаев П.З. \LaTeX~по-русски.
\bibitem{book03} Чебарыков М.С Основы работы в системе \LaTeX.
\end{thebibliography}
\end{document}